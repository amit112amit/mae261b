\documentclass[../main.tex]{subfiles}
\begin{document}
\lstset{language=Matlab}
\chapter{Problem 1}
\section{Introduction} Our objective in this problem is to calculate
the deformation gradient, right Cauchy-Green strain tensor and the
Green strain tensor from a given deformed configuration map.  The
problem gives position map for a uniaxial deformation of a
cylinder. Cylindrical coordinates are our natural choice of
curvilinear coordinates for this problem. The cylindrical basis
vectors in the lab-frame components are given as
\begin{align*}
  \left[\theta_1,\theta_2,\theta_3\right]=\left[\mathbf{e}_R,\mathbf{e}_{\phi},\mathbf{e}_Z\right] =
  \begin{pmatrix}
    \cos(\phi) & -\sin(\phi) & 0\\
    \sin(\phi) & \cos(\phi) & 0\\
    0 & 0 & 1
  \end{pmatrix}
\end{align*}
The reference configuration and the deformed configuration of this
problem are independent of $\phi$ due to the axisymmetric nature of
the problem. The reference configuration is given by
\begin{align*}
  \mathbf{X} = R\mathbf{e}_R + Z\mathbf{e}_Z\quad a_0<R<b_0
\end{align*}
The deformed configuration is given as
\begin{align*}
  \mathbf{x} = \lambda_1R\mathbf{e}_R+\lambda_2Z\mathbf{e}_Z\quad \lambda_1,\lambda_2 > 0
\end{align*}
Now, determinant of deformation gradient for the given reference and
deformed configuration is found to be
$J = |\mathbf{F}| = \lambda_1^2\lambda_2$. A valid deformation should
not invert the volume of the body. Hence $\lambda_1^2\lambda_2 > 0$.

The curvilinear tangent basis vectors are defined as follows:
\begin{align*}
  \mathbf{G}_i &= \frac{\partial\mathbf{X}}{\partial\theta_i}\\
  \mathbf{g}_i &=\frac{\partial\mathbf{x}}{\partial\theta_i}
\end{align*}
The dual basis vectors can be derived from the tangent basis vectors
using the following Kronecker-delta properties
\begin{align*}
  \mathbf{G}_i\cdot\mathbf{G}^j &= \delta_i^j\\
  \mathbf{g}_i\cdot\mathbf{g}^j &= \delta_i^j\\
\end{align*}
The deformation gradient can be computed as
\[\mathbf{F} = \mathbf{g}_i\otimes\mathbf{G}^i\]
Further, the right Cauchy-Green deformation tensor is
\[\mathbf{C}=\mathbf{F}^T\mathbf{F} \]
The Green strain is given as
\[\mathbf{E}= \frac{1}{2}(\mathbf{C}-\mathbf{I})\]

\section{Formulation of Numerical Methods}
We have used Matlab$^\copyright$ for coding this problem. The code for
solving this problem starts with clearing the workspace, command
window and closing any open figure windows. Next, we initialize the
parameters required for numerical calculations --
$a_0,\lambda_1,\lambda_2,\phi$ and $R$.
\begin{lstlisting}[style=Matlab-editor]
  % Clean the set-up
  clear; close all; clc;
  
  % Initialize parameters
  a0 = 1; lambda1 = 1; lambda2 = 2; R = a0; phi = pi/4;
\end{lstlisting}
The next step is to create basis vectors in the lab-frame and
curvilinear co-ordinates as obtained by hand calculations. Note that
the curvilinear basis vectors have been stored as a $3\times3$ matrix
where each column represents a basis vector.
\begin{lstlisting}[style=Matlab-editor]
  % The cylindrical basis vectors
  e_R = [cos(phi), sin(phi), 0]'; e_phi = [-sin(phi), cos(phi), 0]';
  e_Z = [0,0,1]';
  
  % The basis vectors in reference configurations - Gt for tangent
  % basis vectors and Gd for dual basis vectors
  Gt =[e_R, R*e_phi, e_Z]; Gd =[e_R, e_phi/R,e_Z];
  
  % The basis vectors in reference configurations - gt for tangent
  % basis vectors and gd for dual basis vectors
  gt = [lambda1*e_R,lambda1*R*e_phi, lambda2*e_Z]; gd =
  [(1/lambda1)*e_R,(1/(lambda1*R))*e_phi, (1/lambda2)*e_Z];
\end{lstlisting}

The deformation gradient and the two strain tensors can be calculated
using the equations we saw earlier
\begin{lstlisting}[style=Matlab-editor]
  % Deformation Gradient, $ F = gt(:,i) \otimes Gd(:,i) $
  F = gt(:,1)*Gd(:,1)' + gt(:,2)*Gd(:,2)' + gt(:,3)*Gd(:,3)';
  
  % The right Cauchy-Green deformation tensor
  C = F'*F;
  
  % The Green strain
  E = (C - eye(3))/2;
\end{lstlisting}
At last, we will display the calculated arrays and matrices on the
output window
\begin{lstlisting}[style=Matlab-editor]
  display(gt); display(gd); display(Gt); display(Gd); display(F);
  display(C); display(E);
\end{lstlisting}

\section{Calculations and Results}


Using $R = a_0 = 1$, $\lambda_1 = 1$, $\lambda_2 = 2$, $\phi = \pi/4$

% first column
\begin{minipage}[t]{0.5\textwidth}
  \textbf{Analytical Results:}\\
	
  %%%% Tangent basis vectors in reference config %%%%
  The tangent basis vectors
  \begin{align*}
    \left(\mathbf{G}_1,\mathbf{G}_2,\mathbf{G}_3\right) &=
                                                          \begin{pmatrix}
                                                            \cos(\phi) & -R\sin(\phi) & 0\\
                                                            \sin(\phi) &  R\cos(\phi)& 0\\
                                                            0 & 0 & 1\\
                                                          \end{pmatrix}\\
                                                        &\approx
                                                          \begin{pmatrix}
                                                            0.7071 & -0.7071 & 0\\
                                                            0.7071 &  0.7071 & 0\\
                                                            0& 0 & 1\\
                                                          \end{pmatrix}
  \end{align*}
  %%%% Dual basis vectors in reference config %%%%
  The dual basis vectors
  \begin{align*}
    \left(\mathbf{G}^1,\mathbf{G}^2,\mathbf{G}^3\right) &=
                                                          \begin{pmatrix}
                                                            \cos(\phi) & -\sin(\phi)/R & 0\\
                                                            \sin(\phi) &  \cos(\phi)/R& 0\\
                                                            0 & 0 & 1\\
                                                          \end{pmatrix}\\
                                                        &\approx
                                                          \begin{pmatrix}
                                                            0.7071 & -0.7071 & 0\\
                                                            0.7071 &  0.7071 & 0\\
                                                            0	& 0 & 1\\
                                                          \end{pmatrix}
  \end{align*}
  %%%% The tangent basis vectors in DEFORMED config %%%%
  The tangent basis vectors
  \begin{align*}
    \left(\mathbf{g}_1,\mathbf{g}_2,\mathbf{g}_3\right) &=
                                                          \begin{pmatrix}
                                                            \lambda_1\cos(\phi) & -\lambda_1R\sin(\phi) & 0\\
                                                            \lambda_1\sin(\phi) &  \lambda_1R\cos(\phi) & 0\\
                                                            0 & 0 & \lambda_2\\
                                                          \end{pmatrix}\\
                                                        &\approx
                                                          \begin{pmatrix}
                                                            0.7071 & -0.7071 & 0\\
                                                            0.7071 &  0.7071 & 0\\
                                                            0 & 0 & 2\\
                                                          \end{pmatrix}
  \end{align*}
  %%%% Dual basis vectors in DEFORMED config %%%%
  The dual basis vectors
  \begin{align*}
    \left(\mathbf{g}^1,\mathbf{g}^2,\mathbf{g}^3\right) &=
                                                          \begin{pmatrix}
                                                            1/\lambda_1 & -\sin(\phi)/\lambda_1R & 0\\
                                                            0 & \cos(\phi)/\lambda_1R & 0\\
                                                            0 & 0 & 1/\lambda_2\\
                                                          \end{pmatrix}\\
                                                        &\approx
                                                          \begin{pmatrix}
                                                            0.7071 & -0.7071 & 0\\
                                                            0.7071 &  0.7071 & 0\\
                                                            0 & 0 & 0.5\\
                                                          \end{pmatrix}
  \end{align*}
\end{minipage}
% second column
\begin{minipage}[t]{0.5\textwidth}
  \textbf{Numerical Results:}\\
		
  %%%	The tangent basis vectors in reference configuration %%%
  The tangent basis vectors
  \[
  \left(\mathbf{G}_1,\mathbf{G}_2,\mathbf{G}_3\right)=
  \begin{pmatrix}
    0.7071 & -0.7071 & 0\\
    0.7071 &  0.7071 & 0\\
    0      &  0      & 1\\
  \end{pmatrix}
  \]
  \vspace{13mm}
	
  %%%	The dual basis vectors in reference configuration %%%
  The dual basis vectors
  \[
  \left(\mathbf{G}^1,\mathbf{G}^2,\mathbf{G}^3\right)=
  \begin{pmatrix}
    0.7071 & -0.7071 & 0\\
    0.7071 &  0.7071 & 0\\
    0      &  0      & 1\\
  \end{pmatrix}
  \]
  \vspace{13mm}
	
  %%%	The tangent basis vectors in DEFORMED configuration %%%
  The tangent basis vectors
  \[
  \left(\mathbf{g}_1,\mathbf{g}_2,\mathbf{g}_3\right)=
  \begin{pmatrix}
    0.7071 & -0.7071 & 0\\
    0.7071 &  0.7071 & 0\\
    0      &  0      & 2.0\\
  \end{pmatrix}
  \]
  \vspace{13mm}
	
  %%%	The dual basis vectors in DEFORMED configuration %%%
  The dual basis vectors
  \[
  \left(\mathbf{g}^1,\mathbf{g}^2,\mathbf{g}^3\right)=
  \begin{pmatrix}
    0.7071 & -0.7071 & 0\\
    0.7071 &  0.7071 & 0\\
    0      &  0      & 0.5\\
  \end{pmatrix}
  \]
  \vspace{13mm}
\end{minipage}
	
	\begin{minipage}[t]{0.5\textwidth}
          \textbf{Analytical Results:}\\
	
          %%%% Deformation Gradient %%%%
          The deformation gradient
          \begin{align*}
            \mathbf{F} &=
                         \begin{pmatrix}
                           \lambda_1 & 0  & 0\\
                           0 & \lambda_1 & 0\\
                           0  &  0  & \lambda_2\\
                         \end{pmatrix}\\ &\approx
                                           \begin{pmatrix}
                                             1 & 0 & 0\\
                                             0 & 1 & 0\\
                                             0 & 0 & 2\\
                                           \end{pmatrix}
          \end{align*}
          %%%% Right Cauchy-Green Strain %%%%
          The right Cauchy-Green Strain tensor
          \begin{align*}
            \mathbf{C} &=
                         \begin{pmatrix}
                           \lambda_1 & 0 & 0\\
                           0 & \lambda_1^2 & 0\\
                           0 & 0 & \lambda_2^2\\
                         \end{pmatrix}\\ &\approx
                                           \begin{pmatrix}
                                             1 & 0 & 0\\
                                             0 & 1 & 0\\
                                             0 & 0 & 4\\
                                           \end{pmatrix}
          \end{align*}
          %%%% Green Strain %%%%
          The Green Strain tensor
          \begin{align*}
            \mathbf{E} &=
                         \begin{pmatrix}
                           (\lambda_1-1)/2 & 0 & 0\\
                           0 & (\lambda_1^2-1)/2 & 0\\
                           0 & 0 & (\lambda_2^2-1)/2\\
                         \end{pmatrix}\\ &\approx
                                           \begin{pmatrix}
                                             0 & 0 & 0\\
                                             0 & 0 & 0\\
                                             0 & 0 & 1.5\\
                                           \end{pmatrix}
          \end{align*}
	\end{minipage}
	% second column
	\begin{minipage}[t]{0.5\textwidth}
          \textbf{Numerical Results:}\\
	
          %%%	The Deformation Gradient %%%
          The deformation gradient
          \[
          \mathbf{F}=
          \begin{pmatrix}
            1 & 0 & 0\\
            0 & 1 & 0\\
            0 & 0 & 2\\
          \end{pmatrix}
          \]
          \vspace{13mm}
	
          %%%	The right Cauchy-Green Strain %%%
          The right Cauchy-Green Strain
          \[
          \mathbf{C}=
          \begin{pmatrix}
            1 & 0 & 0\\
            0 & 1 & 0\\
            0 & 0 & 4\\
          \end{pmatrix}
          \]\vspace{13mm}
	
          %%%	The Green Strain %%%
          The Green Strain Tensor
          \[
          \mathbf{E}=
          \begin{pmatrix}
            0 & 0 & 0\\
            0 & 0 & 0\\
            0 & 0 & 1.5\\
          \end{pmatrix}
          \]
	\end{minipage}
        \section{Discussion and Conclusions}
        Looking at the equations for reference configuration and deformed configurations
        \begin{align*}
          \mathbf{X} &= R\mathbf{e}_R + Z\mathbf{e}_Z\quad a_0<R<b_0\\
          \mathbf{x} &= \lambda_1R\mathbf{e}_R+\lambda_2Z\mathbf{e}_Z\quad \lambda_1,\lambda_2 > 0
        \end{align*}
        we can see that the deformed configuration is simply an axisymmetric radial scaling of reference 
        configuration along with a scaling in the $\mathbf{e}_Z$ direction. Due to axisymmetry we expect
        \[\mathbf{F}_{11} = \mathbf{F}_{22}\]
        Also, because $\lambda_1$ is the stretching ratio we expect 
        \[\mathbf{F}_{11}=\mathbf{F}_{22}=\lambda_1\]
        The stretching ratio along $\mathbf{e}_Z$ is clearly $\lambda_2$ therefore we expect
        \[\mathbf{F}_{33}=\lambda_2\]
        Our expectation is confirmed by both the hand calculations and numerical results because we obtain
        \begin{align*}
          \mathbf{F} &=
                       \begin{pmatrix}
                         \lambda_1 & 0  & 0\\
                         0 & \lambda_1 & 0\\
                         0  &  0  & \lambda_2\\
                       \end{pmatrix}\\ &\approx
                                         \begin{pmatrix}
                                           1 & 0 & 0\\
                                           0 & 1 & 0\\
                                           0 & 0 & 2\\
                                         \end{pmatrix}
        \end{align*}
        for the chosen values $\lambda_1 = 1$, $\lambda_2 = 2$.
        Physically, the right-Cauchy strain tensor gives the square of local change in distances on deformation
        \[d\mathbf{x} = \mathbf{X}\cdot\mathbf{C}\mathbf{X}\]
        This is consistent with our results
        \begin{align*}
          \mathbf{C} &=
                       \begin{pmatrix}
                         \lambda_1 & 0 & 0\\
                         0 & \lambda_1^2 & 0\\
                         0 & 0 & \lambda_2^2\\
                       \end{pmatrix}\\ &\approx
                                         \begin{pmatrix}
                                             1 & 0 & 0\\
                                             0 & 1 & 0\\
                                             0 & 0 & 4\\
                                           \end{pmatrix}
        \end{align*}
        
        The Green-strain indicates how much $\mathbf{C}$ differs from $\mathbf{I}$ where strain $\mathbf{I}$ means 
        that reference and deformed configurations are the same. Once again, this is seen in our results.
        \begin{align*}
          \mathbf{E} &=
                       \begin{pmatrix}
                         (\lambda_1-1)/2 & 0 & 0\\
                         0 & (\lambda_1^2-1)/2 & 0\\
                         0 & 0 & (\lambda_2^2-1)/2\\
                       \end{pmatrix}\\ &\approx
                                         \begin{pmatrix}
                                           0 & 0 & 0\\
                                           0 & 0 & 0\\
                                           0 & 0 & 1.5\\
                                         \end{pmatrix}
        \end{align*}
        As we see, because the radial stretch has been chosen to be $\lambda_1=1$, which means no change in
        the radial direction, in the Green-strain tensor, the $(1,1)$ and $(2,2)$ components are $0$.
        
        Thus, we find that our calculations and results are consistent with physical expectations.
        \section{Source Code Listing}
        The files related to Problem 1 are
        \begin{enumerate}
        \item HW1\_1.m
        \end{enumerate}
      \end{document}

%%% Local Variables:
%%% mode: latex
%%% TeX-master: "../main"
%%% End:
